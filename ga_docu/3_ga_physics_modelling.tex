\subsection{Physics modelling}
\label{physics_modelling}

The key idea is to represent physical entities and statements using objects that are
represented as directly as possible in the underlying algebra and have a geometric
interpretation. \\

$EGA$ is limited to objects that contain the origin. It can be used to directly model
vectors (i.e. directions), lines and planes through the orgin. In addition, it can model
complex numbers, quaternions and thus rotations in $2d$ and $3d$ well. To realize them it
uses sandwich products for calculating the transformation resulting from two consequtive
reflections either across lines ($2d$-case) or across planes ($3d$-case). The resulting
transformations are rotations. The advantage of using geometric algebra to model those
rotations is the intuitive modelling, due to geometric meaning of the objects used and the
fact that the sandwich products can be used to transform different objects using the same
formulas regardless of the type of object to be transformed.\\

In additon, $PGA$ is capable to model objects that do not contain the origin. Thus it can
be used to directly model points, directions, lines, and planes that may or may not
contain the origin. The additional degree of freedom is possible due to embedding the
Euclidean space of dimension $n$ into a projective space of dimension $n+1$, e.g. $2d$
Euclidean algebra is modelled by embedding it in a $3d$ projective space, and $3d$
Euclidean algebra is modelled in a $4d$ projective space. In a projective space the
objects must be projected into the modelled space. Thus the objects of the embedding space
typically have one dimension more than the corresponding objects in modelled space. E.g. a
projective point (a scalar or $0d$ object) is a vector in the embedding space (a $1d$
object), and a bivector in projective space (a $2d$ object) is used to model a line (a
$1d$ object). The projection reduces the object dimension by one. Projective objects
typcially can be multiplied by a scalar without changing the geometrical meaninig of the
projective object in the embedding space. \\

Due to its extended expressiveness $PGA$ will be used to model physics of mass points and
rigid bodies. $PGA$ will be applied to model mass points, as well as velocity,
acceleration, force, torque, momentum, angular momentum, etc. in order to describe
kinematics and dynamics of mass points and rigid bodies. The approach used here is
inspired by \cite{Plane-based_PGA_Dorst-DeKennik:2022} and
\cite{Dynamics_in_PGA_plane-based_Dorst-DeKennik:2023}, but it does not use plane-based
$PGA$ with planes as vectors, which are based on dual representations of the objects to be
modelled. It rather uses the approach as described by \cite{Lengyel_pga-illuminated:2024}
using projective geometric algebra based on direct representations of points, lines and
planes. The reasons for choosing the latter approach are explained in
\cite{Lengyel_poor-foundations_GA:2024}. Further hints on inconsistencies in GA
application up-to-now can be found as well in \cite{Kritchevsky_case_against_GA:2024}. The
goal of the this work is to be as consistent in terms of mathematical application of GA
and as intuitively applicable as possible. \\









\newpage