\documentclass[12pt,oneside,a4paper]{article}
% Umlaute direkt nutzen (Text muss direkt in utf8 vorliegen => Default bei sublime): statt
% \"A, \"O, \"U, \"a, \"o, \"u und \ss{} direkt Ä, Ö, Ü, ä, ö, ü und ß im Text engeben
\usepackage[utf8]{inputenc}  % Umlaute direkt eingeben (ggf. latin1 statt utf8)
\usepackage[T1]{fontenc}     % Ausgabe der Umlaute
\usepackage{lmodern}

\usepackage[dvips]{graphicx}       % alles zum Grafiken einbinden
\graphicspath{ {./images/} }       % pointer to subdir for all images

\usepackage{caption}
\usepackage{subcaption}

\usepackage{eurosym}               % Eurosymbol

\usepackage[centertags]{amsmath}   % Mathekram: amstex
\usepackage{amsfonts}
\usepackage{amssymb}
\usepackage{cases}
\usepackage{latexsym}              % math symbols
\usepackage{exscale}               % Summen-/Integralzeichen in richtiger Groesse
\usepackage{MnSymbol}              % MnSymbols, e.g. \wedgedot, \veedot

\usepackage[dvipsnames]{xcolor}    % Textcolor for dvips w/ standard color names
\usepackage{fancyhdr}              % to change header and footers

% Deutsch als Hauptsprache im Dokument (Layout von Datum, Trennungsregeln etc.)
% Umschaltbar mit \selectlanguage{}
\usepackage[main=english,ngerman]{babel}  % support "Neue Rechtschreibung"

\usepackage[version=4]{mhchem}     % Chemische Formeln (muss nach amsmath kommen!)

\usepackage{bookmark}
\usepackage{hyperref}  % must be imported as last package!
\hypersetup{
    colorlinks=true,
    linkcolor=blue,
    filecolor=magenta,      
    urlcolor=cyan,
    pdftitle={Overleaf Example},
    pdfpagemode=FullScreen,
    }

\usepackage{scalerel}
\usepackage{accents}

%\DeclareMathSizes{11}{19}{13}{9}
%\DeclareMathSizes{12}{20}{14}{10}  % Larger formulas vs. text

\title{Geometric Algebra (GA)}
\author{Daniel Hug}
\date{April 2025}


\pagestyle{fancy}       % Turn on the style
\fancyfoot{}            % Clear the footer
\fancyfoot[R]{\thepage} % Set the right side of the footer to be the page numberå

% new commands for partial derivatives
\newcommand{\pd}[2]{\ensuremath{\frac{\partial {#1}}{\partial {#2}}}}

% left and right contraction
\newcommand*{\lcontr}{\ensuremath{\rfloor}}
\newcommand*{\rcontr}{\ensuremath{\lfloor}}

% grade projection operator
\newcommand{\grprj}[2]{\ensuremath{\langle {#1} \rangle_{#2}}}

% norms
\newcommand{\nrm}[1]{\left \lVert {#1} \right \rVert}

% bulk and weight (by adding a filled or empty circle as right downstairs index)
\newcommand{\bulk}[1]{\ensuremath{{#1}_\bullet}}
\newcommand{\weight}[1]{\ensuremath{{#1}_\circ}}

% parallel sign scaled to the hight of the perpendicular sign (package scalerel)
\newcommand*{\parall}{\stretchrel*{\parallel}{\perp}}

% underbar by using \bar with the accents package
\newcommand{\ubar}[1]{\underaccent{\bar}{#1}}


\begin{document}

\maketitle

% no paragraph indentation on first line
\setlength{\parindent}{0pt}

\section{Documentation}

Overview (high-level initial intros on youtube):
\scriptsize
\begin{itemize}
    \item \href{https://www.youtube.com/watch?v=1cRFfYQYGxE&list=PLcGKfGEEONaBNsY_bOj8IhbCPvj6OAdWo&index=33}{\textit{https://www.thestrangeloop.com}}
    \item \href{https://www.youtube.com/watch?v=60z_hpEAtD8&t=17s}{\textit{https://www.youtube.com/@sudgylacmoe}}
    \item \href{https://www.youtube.com/watch?v=zgi-13F2Kec&list=PLsSPBzvBkYjxV8QMlOTYgrqa-nmlt_L7A&index=11}{\textit{https://www.youtube.com/@bivector}}
\end{itemize}

\normalsize

Sources:
\scriptsize
\begin{itemize}
    \item \url{https://projectivegeometricalgebra.org}
    \item \url{https://terathon.com/blog/poor-foundations-ga.html}
    \item \url{https://bivector.net}
    \item \url{https://www.youtube.com/@bivector}
    \item \url{https://www.youtube.com/@sudgylacmoe}
    \item \url{https://www.youtube.com/watch?v=ItGlUbFBFfc}
    \item \url{https://www.youtube.com/watch?v=v-WG02ILMXA&t=1476s}
\end{itemize}
\normalsize


\newpage

\subsection{Introduction}
\label{intro}

The ga library is intended to make Geometric Algebra more accessible by providing
functionality for numerical calculations required for simulation of physical systems.
\\

It implements functionality for two-dimensional $(2d)$ and three-dimensional $(3d)$ spaces
with Euclidean geometry. Currently it provides data types and operations for
non-projective Euclidean geometry $(ega2d, ega3d)$ and projective Euclidean geometry
$(pga2dp, pga3dp)$. The library handles multivectors and their components efficiently. It
can be easily extended to provide functionality for handling spacetime algebra $(STA)$
with limited effort.
\\

The main library is a header only library in the ga folder which can be used by including
either
\begin{verbatim}
    #include ga/ga_ega.hpp // or
    #include ga/ga_pga.hpp
\end{verbatim}
and by making the corresponding namespaces accessible for use by stating
\begin{verbatim}
    using namespace hd::ga;      // and either
    using namespace hd::ga::ega; // or
    using namespace hd::ga::pga;
\end{verbatim}
in your application code. For a simple usage example please have a look either at
\path{ga_test/src/ga_ega_test.cpp} or at \path{ga_test/src/ga_pga_test.cpp}.
\\

All product expressions are generated by the file \path{ga_prdxpr/ga_prdxpr_main.cpp} and
user input provided in the header files of the corresponding algebra. The generated
coefficient expressions are used to fill-in the source code for geometric products, wedge
and dot products or their regressive counterparts. If certain combinations are not yet
implemented in the ga library they can easily be generated within \path{ga_prdxpr} as
needed and then be added to the library. \\

The complement operation used in this library corresponds to the complement as defined in
\cite{Lengyel_pga-illuminated:2024}. It is uniquely determined with respect to the outer
product (not with respect to the geometric product as commonly used by many autors
actively working on geometric algebra). This is helpful to generate consistent signs of
expressions, operators, complements and duals. The left complement $\ubar{u}$ is defined
as $\ubar{u} \wedge u = I_n$ and the right complement $\bar{u}$ as $u \wedge \bar{u} =
I_n$ with $I_n$ as the pseudoscalar of the $n$-dimensional space modeled by the algebra.
\\

The complement operation is used in turn to define a unique dualization operation that
works for cases when the metric is non-degenerate as well as for cases where it is
degenerate, like in projective geometric algebra. The right dual $A^{\star}$ is defined as
$A^{\star} = \overline{GA}$, where $A$ is an arbitrary multivector and $G$ the extended
metric. There is also a corresponding left dualization operation defined as $A_{\star} =
\underline{GA}$. $\star$ is used as dualization operator (Hodge star). Relations to the
geometric product $A^{\star} = A^{\dagger} I_n$ and $A_{\star} = I_n A^{\dagger}$ also
exist with $\dagger$ as the reversion operation. \\


TODO: \emph{fill in further introductory notes here}

\newpage

\subsection{Basic formulas}
\label{basic_formulas}

The following is provided for Euclidean algebra of two-dimensional space (\emph{ega2d})
using an orthormal basis \bv{1}, \bv{2}:
\begin{subequations}
    \begin{align}
        (\bv{1})^2 & = \bv{1}^2 = +1, \;\text{and}\; (\bv{2})^2 = \bv{2}^2 = +1
        \label{eq:ega2d_base_vec} \\
        s_{2d} & = s\bm{1}
        \label{eq:scalar2d} \\
        v_{2d} & = v_1\bv{1} + v_2\bv{2} 
        \label{eq:vec2d} \\
        ps_{2d} & = ps \bv{12}  = ps\mathds{1}
        \label{eq:pscalar2d}
    \end{align}
\end{subequations}
equation (\ref{eq:scalar2d}) is the scalar part $s_{2d}$, equation (\ref{eq:vec2d})
contains the vector part $v_{2d}$, and equation (\ref{eq:pscalar2d}) contains the
pseudoscalar part $ps_{2d}$. The index $2d$ is typically omitted when clear from context.
The basis elements are $\{\bm{1}, \bv{1}, \bv{2}, \bv{12}\}$. Using these components a
multivector $M$ of $2d$ space is defined as
\begin{equation}
     M = s \bm{1} + v_1\bv{1} + v_2\bv{2} + ps\bv{12}
    \label{eq:mvec2d}   
\end{equation}
where equation (\ref{eq:mvec2d}) contains three parts: the scalar part $s\bm{1}$ (basis
element is the scalar \textbf{1}; if \textbf{1} is not shown, it is implicitly assumed for
scalar values), the vector part $v$ (basis elements \bv{1} and \bv{2}) and the
pseudoscalar part $ps \bv{12} = ps \mathds{1}$ (basis element is \bv{12}, which is
sometimes written as $\mathds{1}$ to show its character as pseudoscalar of this space. It
is a bivector in $2d$-Euclidean space). \\

For Euclidean algebra of three-dimensional space (\emph{ega3d}) using an orthonormal basis
\bv{1}, \bv{2}, \bv{3} there are:
\begin{subequations}
    \begin{align}
        (\bv{1})^2 = \bv{1}^2 & = +1, (\bv{2})^2 = \bv{2}^2 = +1,
        \;\text{and}\; (\bv{3})^2 = \bv{3}^2 = +1
        \label{eq:ega3d_base_vec} \\ 
        s_{3d} & = s\bm{1}
        \label{eq:scalar3d} \\
        v_{3d} & = v_1\bv{1} + v_2\bv{2} + v_3\bv{3} 
        \label{eq:vec3d} \\ 
        b_{3d} & = b_1\bv{23} + b_2\bv{31} + b_3\bv{12} 
        \label{eq:bivec3d} \\ 
        ps_{3d} & = ps \bv{123}  = ps\mathds{1}
        \label{eq:pscalar3d}
    \end{align}
\end{subequations}
equation (\ref{eq:scalar3d}) is the scalar part $s_{3d}$, equation (\ref{eq:vec3d}) is the
vector part $v_{3d}$, equation (\ref{eq:bivec3d}) is the bivector part $b_{3d}$, and
equation (\ref{eq:pscalar3d}) is the pseudoscalar part $ps_{3d}$. The index $3d$ is
typically omitted when clear from context. Comparing to the $2d$-case it becomes obvious,
that all parts depend on context, specifically on the dimensionality of the modeled space,
and thus need to be defined and treated accordingly (\emph{hint}: since \Cpp is a
statically typed language those types need to be well-defined and distinguishable from
each other). The basis elements are $\{\bm{1}, \bv{1}, \bv{2}, \bv{3}, \bv{23}, \bv{31},
\bv{12}, \bv{123}\}$. Using these components a multivector $M$ of $3d$ space is defined as
\begin{equation}
    M = s \bm{1} + v_1\bv{1} + v_2\bv{2} + v_3\bv{3} 
    + b_1\bv{23} + b_2\bv{31} + b_3\bv{12} + ps\bv{123}
    \label{eq:mvec3d}  
\end{equation}
where equation (\ref{eq:mvec3d}) contains four parts: the scalar component $s\bm{1}$
(basis element is the scalar \textbf{1}, the vector part $v$ (basis elements \bv{1},
\bv{2} and \bv{3}), the bivector part $b$ (basis elements \bv{23}, \bv{31} and \bv{12})
and the pseudoscalar part $ps \bv{123} = ps \mathds{1}$ (basis element is \bv{123}, which
is sometimes written as $\mathds{1}$ to show its character as pseudoscalar of this space.
It is a trivector in $3d$-Euclidean space). \\

TODO: \emph{fill in basic product definitions of ega2d and ega3d here} \\

TODO: \emph{fill in pga2d and pga3d definitions here} \\


The inner product between multivectors $A$ and $B$ is defined as
\begin{equation}
    A \bullet B = A^{T} G B
    \label{eg:inner_product_metric}
\end{equation}
with $G$ as the extended metric and matrix multiplication on the right hand side. It
satisfies the identity
\begin{equation}
    A \bullet B  = \grprj{A \wedgedot \tilde{B}}{0} = \grprj{ B \wedgedot \tilde{A} }{0}
    \label{eq:inner_product_link_to_gpr}
\end{equation}
where $\wedgedot$ stands for the geometric product within the grade projection operator
\grprj{}{k} for grade $k$. The norm of a multivector $A$ is defined as
\begin{equation}
    \nrm{A} = \sqrt{A \bullet A}
    \label{eq:norm_mv}
\end{equation}
with a positive argument under the square root due to the definition of the inner product.
\\

The contraction is explicitly defined as
\begin{subequations}
    \begin{align}
    A \lcontr B & = A \ll B = A_{\star} \vee B
    \label{eq:lcontr} \\
    B \rcontr A & = B \gg A = B \vee A^{\star}
    \label{eq:rcontr}
    \end{align}
\end{subequations}
with $\star$ denoting the Hodge dual which is formed by the respective complement
operation (in spaces of even dimension the left or right complement respectively, and in
spaces of uneven dimension the complement function regardless of the side of the operand).
For blades A and B, they satisfy
\begin{subequations}
    \begin{align}
    A \lcontr B &  = A \ll B = \grprj{B \wedgedot \tilde{A}}{gr(B)-gr(A)}
    \label{eq:lcontr_gpr} \\
    B \rcontr A & = B \gg A = \grprj{\tilde{A} \wedgedot B}{gr(B)-gr(A)}
    \label{eq:rcontr_gpr}
    \end{align}
\end{subequations}
and fulfill the requirement that $A \lcontr B = A \rcontr B = A \bullet B$ whenever $A$
and $B$ have the same grade. In this case the contractions reduce to the inner product. \\


The geometric product between and vector $v$ and a blade $B$ is defined in terms of
interior and exterior products, i.e. the right contraction $\rcontr$, the left contraction
$\lcontr$ and wedge product $\wedge$ as follows:
\begin{subequations}
    \begin{align}
    a \wedgedot B & =  B \rcontr a + a \wedge B
    \label{eq:gpr_Brc} \\
    B \wedgedot a & =  a \lcontr B + B \wedge a
    \label{eq:gpr_Blc}
    \end{align}
\end{subequations}
or sometimes using the right shift ($\gg$) or left shift ($\ll$) operators directly
\begin{subequations}
    \begin{align}
    a \wedgedot B & = B \gg a + a \wedge B
    \label{eq:gpr_Bshrc} \\
    B \wedgedot a & =  a \ll B + B \wedge a
    \label{eq:gpr_Bshlc}
    \end{align}
\end{subequations}
as an alternative. \\

Every vector $a$ can be decomposed into parts $a_{\parall}$ parallel to the $k$-blade $B$
and $a_{\perp}$ perpendicular to $B$, such that $a = a_{\parall} + a_{\perp}$. For
equation {(\ref{eq:gpr_Bshrc})}: $B \gg a$ is a $(k-1)$-blade. If $a_{\parall} \neq 0$
then $B \gg a$ represents a subspace of $B$. $a \wedge B$ is a $(k+1)$-blade. If
$a_{\perp} \neq 0$ then $a \wedge B$ represents $span(a,B)$.

For arguments of equal grade the contractions reduce to the dot
product $\bullet$, so that one can write specifically for two vectors $a$ and $b$:
\begin{equation}
    a \wedgedot b =  a \bullet b + a \wedge b
\end{equation}

TODO: \emph{fill in further basic formulas here} \\

% Text with a norm $\nrm{x}$ and an indexed norm as $\bulk{\nrm{u}}$ and
% $\weight{\nrm{u}}$.\\
Text with a norm $\nrm{x}$ and an indexed norm as $\bulknrm{u}$ and
$\weightnrm{u}$.\\

$\bm{e}_1$, \bv{1}, $\bv{123}$, $\filledstar$, $\star$, $\smallstar$
\\




\newpage
% literature and references
\subsection{Literature}
\label{literature}

%\addcontentsline{toc}{section}{\numberline{}Literature}

\bibliographystyle{apalike}
\bibliography{9_ga_literature-db}

\end{document}