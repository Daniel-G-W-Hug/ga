\subsection{Introduction}

The ga library is intended to make Geometric Algebra more accessible by providing
functionality for numerical calculations required for simulation of physical systems. It
implements functionality for two-dimensional $(2D)$ and three-dimensional $(3D)$ spaces
with Euclidean geometry. Currently it provides data types and operations for
non-projective Euclidean geometry $(ega2d, ega3d)$ and projective Euclidean geometry
$(pga2dp, pga3dp)$. The library handles multivectors and their components efficiently. It
could be easily extended to provide functionality for handling spacetime algebra $(STA)$
with limited effort. \\

The main library is a header only library in the ga folder which can be used by including
either
\begin{verbatim}
    #include ga/ga_ega.hpp // or
    #include ga/ga_pga.hpp
\end{verbatim}
and by making the corresponding namespaces accessible for use by stating
\begin{verbatim}
    using namespace hd::ga;      // and either
    using namespace hd::ga::ega; // or
    using namespace hd::ga::pga;
\end{verbatim}
in your application code. For a simple usage example please have a look either at
\path{ga_test/src/ga_ega_test.cpp} or at \path{ga_test/src/ga_pga_test.cpp}. \\

All product expressions are generated by the file \path{ga_prdxpr/ga_prdxpr_main.cpp} and
user input provided in the header files of the corresponding algebra. The generated
coefficient expressions are used to fill-in the source code for geometric products, wedge
and dot products. If certain combinations are not yet implemented in the ga library they
can easily be generated within \path{ga_prdxpr} as needed and then be added to the
library. \\



TODO: \emph{fill in further introductory notes here}

\newpage
