\subsection{Introduction}
\label{intro}

The ga library is intended to make Geometric Algebra more accessible by providing
functionality for numerical calculations required for simulation of physical systems.
\\

It implements functionality for two-dimensional $(2D)$ and three-dimensional $(3D)$ spaces
with Euclidean geometry. Currently it provides data types and operations for
non-projective Euclidean geometry $(ega2d, ega3d)$ and projective Euclidean geometry
$(pga2dp, pga3dp)$. The library handles multivectors and their components efficiently. It
could be easily extended to provide functionality for handling spacetime algebra $(STA)$
with limited effort.
\\

The main library is a header only library in the ga folder which can be used by including
either
\begin{verbatim}
    #include ga/ga_ega.hpp // or
    #include ga/ga_pga.hpp
\end{verbatim}
and by making the corresponding namespaces accessible for use by stating
\begin{verbatim}
    using namespace hd::ga;      // and either
    using namespace hd::ga::ega; // or
    using namespace hd::ga::pga;
\end{verbatim}
in your application code. For a simple usage example please have a look either at
\path{ga_test/src/ga_ega_test.cpp} or at \path{ga_test/src/ga_pga_test.cpp}.
\\

All product expressions are generated by the file \path{ga_prdxpr/ga_prdxpr_main.cpp} and
user input provided in the header files of the corresponding algebra. The generated
coefficient expressions are used to fill-in the source code for geometric products, wedge
and dot products. If certain combinations are not yet implemented in the ga library they
can easily be generated within \path{ga_prdxpr} as needed and then be added to the
library.
\\

The complement operation used in this library corresponds to the complement as defined in
\cite{Lengyel_pga-illuminated:2024}. It is uniquely determined with respect to the outer
product (not with respect to the geometric product as commonly used by many autors). The
left complement $\ubar{u}$ is defined as $\ubar{u} \wedge u = I_n$ and the right
complement $\bar{u}$ as $u \wedge \bar{u} = I_n$ with $I_n$ as the pseudoscalar of the
$n$-dimensional space modeled by the algebra.
\\

The complement operation is used in turn to define a unique dualization operation that
works for cases when the metric is non-degenerate as well as for cases where it is
degenerate, like in projective geometric algebra. The right dual $A^{\star}$ is defined as
$A^{\star} = \overline{GA}$, where $A$ is an arbitrary multivector and $G$ the extended
metric. There is also a corresponding left dualization operation defined as $A_{\star} =
\underline{GA}$. $\star$ is used as dualization operator (Hodge star). Relations to the
geometric product $A^{\star} = A^{\dagger} I_n$ and $A_{\star} = I_n A^{\dagger}$ also
exist with $\dagger$ as the reversion operation. \\


TODO: \emph{fill in further introductory notes here}

\newpage
