\subsection{Introduction}
\label{intro}

The ga library is intended to make Geometric Algebra (GA) more accessible by providing
functionality for numerical calculations required for simulation of physical systems. \\

It implements functionality for two- and three-dimensional spaces with Euclidean geometry.
Currently it provides data types and operations to handle the algebras of regular
Euclidean geometry $(ega2d, ega3d)$ and projective Euclidean geometry $(pga2dp, pga3dp)$.
The library handles fully populated multivectors, even and odd grade multivectors as well
as their components like scalars, vectors, bivectors, trivectors, and pseudoscalars of the
corresponding linear spaces efficiently. It also provides operations to work with these
objects. It was designed with extensibiliy in mind and could be easily extended to provide
functionality for handling additional algebras, e.g. spacetime algebra $(STA)$, with
limited effort. \\

The main library is a header only library in the ga sub-folder of the library which can be
used by including either
\begin{verbatim}
    #include ga/ga_ega.hpp // or
    #include ga/ga_pga.hpp
\end{verbatim}
and by making the corresponding namespaces accessible by stating
\begin{verbatim}
    using namespace hd::ga;      // and either
    using namespace hd::ga::ega; // or
    using namespace hd::ga::pga;
\end{verbatim}
in your application code. For a simple usage example please have a look either at
\path{ga_test/src/ga_ega_test.cpp} or at \path{ga_test/src/ga_pga_test.cpp}.
\\

All product expressions for inner, outer, interior and geometric products and their
regressive counterparts are generated via the file \path{ga_prdxpr/ga_prdxpr_main.cpp} and
user input provided in the header files of the corresponding algebra in that folder. The
generated coefficient expressions are used in the source code to implement dot products,
wedge products, contractions, expansions, geometric products or their regressive
counterparts, as well as sandwich products describing rotors and motors of the
corresponding algebras. If certain products are required for your applicaton, but not yet
implemented in the ga library they can easily be generated within
\path{ga_prdxpr/src_prdxpr} as needed and added with minimal effort. \\

As mentioned before, we distinguish inner, interior, outer and geometric products.
\begin{itemize}
    \item Inner and interior products are grade reducing operations.
    \item The inner product combines arguments of the same type and grade, while the
    interior products involve complements/duals in at least one of their arguments,
    combining and linking arguments of different grades. Inner products are linked to the
    metric of the modelled space. 
    \item The outer product is a grade-increasing operation adding the grades of its
    arguments. Outer products, like the wedge product, connect subspaces represented by
    their operands and create new subspaces defined by the span of the subspaces of the
    operands.
    \item Geometric products are the combination of both types of products in one
    expression. They typically result in multivectors consisting of parts with different
    grades as a result of the operation. Their main advantage is invertibility under
    certain circumstances.
    \item Regressive products use a complement operation on their arguments before and/or
    after applying the operation linking the arguments of the product.
    \item Sandwich products are used to implement orthogonal transformations of objects.
    They realize rotations or general motions including rotation and translations. This
    can be achieved by multiplying the object symmetrically from the left and right hand
    side with the appropriate objects.
\end{itemize}

The complement operation used in this library is defined
in~\cite{Lengyel_pga-illuminated:2024}. It is uniquely determined with respect to the
outer product (not with respect to the geometric product, as commonly used for dualization
by many autors working with GA). This helps to generate consistent signs of expressions,
operators, complements and duals. The left complement $\ubar{u}$ and the right complement
$\bar{u}$ are defined as
\begin{subeqnarray}
    % use \label for full label
    % use \slabel for sublabel
    \ubar{u} \wedge u & = & I_n \slabel{eq:lcmpl_def} \\
    u \wedge \bar{u}  & = & I_n \slabel{eq:rcmpl_def}
\end{subeqnarray}
with $I_n  = \bv{1} \wedge \bv{2} \wedge \ldots \wedge \bv{n}$ as the pseudoscalar of the
$n$-dimensional algebra, and the wedge symbol $(\wedge)$ as the operator symbol of the
outer product. The effect of the complement is to turn basis elements not contained in the
object $u$ into the constituting basis elements of the complement of the object $\ubar{u}$
or $\bar{u}$. Connecting the object and its complement with the wedge products creates an
object that fills the available space in the sense of requiring all basis elements of the
space to represent it. The full space itself is represented by the pseudoscalar of that
space, since the pseudoscalar $I_n$ is formed formed by the product of all $n$ basis
vectors of the space. \\

It can be shown that the left and right complements fulfill
\begin{equation}
    \bar{\ubar{a}} = a
    \label{eq:double_cmpl}
\end{equation}
thus the left and right complement operations are inverses to each other. This is valid
independent of the sequence of application, independent of the grade of the argument $a$,
and independent of the dimension of the space modelled in the algebra. Based on this it is
possible to define so-called regressive products using rules analog to deMorgan's rules in
boolean algebra as follows (definitions can be found
in~\cite{Lengyel_pga-illuminated:2024} and in \cite{Browne_Grassmann-Algebra_Vol1:2012}):
\begin{subeqnarray}
    % use \label for full label
    % use \slabel for sublabel
    a \vee b & \equiv & \underline{\bar{a} \wedge \bar{b}} \slabel{eq:rwdg_std} \\
    a \vee b & \equiv & \overline{\ubar{a} \wedge \ubar{b}} \slabel{eq:rwdg_alt}
\end{subeqnarray}
The same complement is applied to both arguments first, the resulting expressions are
combined with the wedge product, and the inverse complement operation is used to transform
the outcome into the final result, the regressive product. The inverted wedge symbol
$(\vee)$ is used for the regressive wedge product. Both definitions~(\ref{eq:rwdg_std})
and~(\ref{eq:rwdg_alt}) are equivalent, but~(\ref{eq:rwdg_std}) is used within the library
to derive expressions for regressive products. \\

There are some additional formulas for complements which are useful for working with GA
expressions and thus are provided here for reference. Taking the right and left
complements of~(\ref{eq:rwdg_std}) and~(\ref{eq:rwdg_alt}) we end up with
\begin{subeqnarray}
    % use \label for full label
    % use \slabel for sublabel
    \overline{a \vee b}  & = & \overline{\underline{\bar{a} \wedge \bar{b}}}
    \quad  =  \quad \bar{a} \wedge \bar{b}
    \slabel{eq:intermediate_std} \\
    \underline{a \vee b} & = & \underline{\overline{\ubar{a} \wedge \ubar{b}}}
    \quad  =  \quad  \ubar{a} \wedge \ubar{b}
    \slabel{eq:intermediate_alt}
\end{subeqnarray}
Substituting $a = \ubar{u}$ and $b = \ubar{v}$ into~(\ref{eq:intermediate_std}) and $a =
\bar{u}$ and $b = \bar{v}$ into~(\ref{eq:intermediate_alt}) and exchanging the left and
right hand sides of the equations we end up with
\begin{subeqnarray}
    % use \label for full label
    % use \slabel for sublabel
    u \wedge v & = & \overline{\ubar{u} \vee \ubar{v}} \slabel{eq:wdg_std} \\
    u \wedge v & = & \underline{\bar{u} \vee \bar{v}} \slabel{eq:wdg_alt}
\end{subeqnarray}
This shows that the outer product and the regressive outer product can be expressed by
each other.

The complement operation is used to define a unique dualization operation that works for
cases where the metric is non-degenerate and even for cases where it is degenerate, like
in projective geometric algebra $PGA$. This is the main advantage compared to the usual
prodedure of dualization by multiplying the object with the pseudoscalar (or its inverse),
since the latter breaks down for spaces with degenerate metrics. The right dual
$A^{\star}$ is defined as $A^{\star} = \overline{GA}$, where $A$ is an arbitrary
multivector, $G$ the extended metric and $GA$ a matrix-vector-product between them.
$A^{\star}$ is also called the Hodge dual. There is also a corresponding left dualization
operation defined as $A_{\star} = \underline{GA}$. $\star$ is used as dualization operator
(Hodge star) in both cases. The difference between the hodge dual and the complement is
taking the metric into account before taking the corresponding complement. However, this
does not lead to different results in spaces where the metric is the identity, but is
relevant for spaces with a more involved metric, e.g. in projective spaces. Relations to
the geometric product are given by
\begin{subeqnarray}
    A^{\star} & = &  A^{\dagger} \wedgedot I_n \slabel{eq:hodge_right} \\
    A_{\star} & = & I_n \wedgedot A^{\dagger} \slabel{eq:hodge_left}
\end{subeqnarray}
with $I_n = \bv{1} \wedge \bv{2} \wedge \ldots \wedge \bv{n}$ as the pseudoscalar of the
$n$-dimensional space and where $\dagger$ is the reversion operation. Following identities
hold for operands of same grade. They link the outer product and the dot product in that
case: 
\begin{subeqnarray}
    A \wedge B^{\star} & = & (A \bullet B) I_n, \quad \text{when \path{gr(A)} =
    \path{gr(B)}} \\
    A_{\star} \wedge B & = & (A \bullet B) I_n, \quad \text{when \path{gr(A)} =
    \path{gr(B)}}
\end{subeqnarray}
The influence of the
metric for the dualization operation can be seen in its effect for the different algebras
which also depends on the dimension of the modelled space:
\begin{itemize}
\item $ega2d: A^{\star} = \text{\path{right_dual(A)}}$, $A_{\star} =
\text{\path{left_dual(A)}}$
\item $ega3d: A^{\star} = \text{\path{dual(A)}}$, $A_{\star} = \text{\path{dual(A)}}$
\item $pga2dp: A^{\star} = \text{\path{bulk_dual(A)}}$, $A_{\star} =
\text{\path{bulk_dual(A)}}$
\item $pga3dp: A^{\star} = \text{\path{right_bulk_dual(A)}}$, $A_{\star} =
\text{\path{left_bulk_dual(A)}}$
\end{itemize}
In even dimensional spaces we have to distinguish left and right operations, while in odd
dimensional spaces we do not have to, and for projective spaces, like $pga$, the
degenerate metric results in using the bulk part only for the dualization operation with
the Hodge star.\\


Following bi-argument operations $\emph{op}(a,b) = a \; op \; b$ and their regressive
variants $\emph{rop}(a,b) = \underline{\bar{a} \; \emph{op} \; \bar{b}}$ are implemented
(with \path{op(a,b)} or \path{rop(a,b)} as the name of the implemented function). Using
$A$ vs. $a$ implies $gr(A) \ge gr(a)$, where $gr(arg)$ returns the grade of the argument
$arg$:
\begin{itemize}
    \item inner product, delivering a scalar: $a \bullet b$, \path{dot(a,b)}
    \item outer product of a $j-$ and $k-$vector, delivering a $j+k$-vector in
    $span(a,b)$: $a \wedge b$, \path{wdg(a,b)}
    \item geometric product:  $a \wedgedot b$, \path{operator*(a,b)}
    \item commutator product (the asymmetric part of the geometric product): $cmt(a,b) =
    \frac{1}{2}(a \wedgedot b - b \wedgedot a)$,
    \path{cmt(a,b)}
    \item regressive inner product, delivering a pseudoscalar: $a \circ b$,
    \path{rdot(a,b)}
    \item regressive outer product: $a \vee b$, \path{rwdg(a,b)}
    \item regressive geometric product: $a \veedot b$, \path{rgpr(a,b)}
    
    \item left (bulk) contraction: $a \lcontr B = a \ll B = a_{\star} \vee B $,
    \path{operator<<(a,B)}, which implements \path{left_bulk_contract(a,B)}  
    \item right (bulk) contraction: $B \rcontr a = B \gg a = B \vee a^{\star} $,
    \path{operator>>(B,a)}, which implements \path{right_bulk_contract(B,a)}

    \item left weight contraction: $a_{\smallstar} \vee B $,
    \path{left_weight_contract(a,B)} 
    \item right weight contraction: $B \vee a^{\smallstar} $,
    \path{right_weight_contract(B,a)} 

    \item left bulk expansion: $A_{\star} \wedge b $,
    \path{left_bulk_expand(A,b)} 
    \item right bulk expansion: $a \wedge B^{\star} $,
    \path{right_bulk_expand(a,B)} 

    \item left weight expansion: $A_{\smallstar} \wedge b $,
    \path{left_weight_expand(A,b)} 
    \item right bulk expansion: $a \wedge B^{\smallstar} $,
    \path{right_weight_expand(a,B)} 
\end{itemize}
where the items from the left contraction downwards are all considered interior products.
They operate on subspaces after applying a dualization operation to one operand. \\


TODO: \emph{fill in further introductory notes here} \\

\newpage
