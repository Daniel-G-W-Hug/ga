\subsection{Introduction}
\label{intro}

The ga library is intended to make Geometric Algebra (GA) more accessible by providing
functionality for numerical calculations required for simulation of physical systems. \\

It implements functionality for two- and three-dimensional spaces with Euclidean geometry.
Currently it provides data types and operations to handle the algebras of regular
Euclidean geometry $(ega2d, ega3d)$ and projective Euclidean geometry $(pga2dp, pga3dp)$.
The library handles fully populated multivectors, even and odd grade multivectors as well
as their components like scalars, vectors, bivectors, trivectors, and pseudoscalars of the
corresponding linear spaces efficiently. It also provides operations to work with these
objects. It was designed with extensibiliy in mind and could be easily extended to provide
functionality for handling additional algebras, e.g. spacetime algebra $(STA)$, with
limited effort. \\

The main library is a header only library in the ga sub-folder of the library which can be
used by including either
\begin{verbatim}
    #include ga/ga_ega.hpp // or
    #include ga/ga_pga.hpp
\end{verbatim}
and by making the corresponding namespaces accessible by stating
\begin{verbatim}
    using namespace hd::ga;      // and either
    using namespace hd::ga::ega; // or
    using namespace hd::ga::pga;
\end{verbatim}
in your application code. For a simple usage example please have a look either at
\path{ga_test/src/ga_ega_test.cpp} or at \path{ga_test/src/ga_pga_test.cpp}.
\\

All product expressions for inner, outer, interior and geometric products and their
regressive counterparts are generated via the file \path{ga_prdxpr/ga_prdxpr_main.cpp} and
user input provided in the header files of the corresponding algebra in that folder. The
generated coefficient expressions are used in the source code to implement dot products,
wedge products, contractions, expansions, geometric products or their regressive
counterparts, as well as sandwich products describing rotors and motors of the
corresponding algebras. If certain products are required for your applicaton, but not yet
implemented in the ga library they can easily be generated within
\path{ga_prdxpr/src_prdxpr} as needed and added with minimal effort. \\

As mentioned before, we distinguish inner, interior, outer and geometric products.
\begin{itemize}
    \item Inner and interior products are grade reducing operations.
    \item The inner product combines arguments of the same type and grade, while the
    interior products involve complements/duals in at least one of their arguments,
    combining and linking arguments of different grades. Inner products are linked to the
    metric of the modelled space. 
    \item The outer product is a grade-increasing operation adding the grades of its
    arguments. Outer products, like the wedge product, connect subspaces represented by
    their operands and create new subspaces defined by the span of the subspaces of the
    operands.
    \item Geometric products are the combination of both types of products in one
    expression. They typically result in multivectors consisting of parts with different
    grades as a result of the operation. Their main advantage is invertibility under
    certain circumstances.
    \item Regressive products use a complement operation on their arguments before and/or
    after applying the operation linking the arguments of the product.
    \item Sandwich products are used to orthogonally transform objects implementation
    operators for rotation or general motion including rotation and translation. This can
    be achieved by multiplying the object symmetrically from the left and right side.
\end{itemize}

The complement operation used in this library is defined
in~\cite{Lengyel_pga-illuminated:2024}. It is uniquely determined with respect to the
outer product (not with respect to the geometric product, as commonly used for the
dualization operation by many autors working with GA). This helps to generate consistent
signs of expressions, operators, complements and duals. The left complement $\ubar{u}$ and
the right complement $\bar{u}$ are defined as
\begin{subeqnarray}
    % use \label for full label
    % use \slabel for sublabel
    \ubar{u} \wedge u & = & I_n \slabel{eq:lcmpl_def} \\
    u \wedge \bar{u}  & = & I_n \slabel{eq:rcmpl_def}
\end{subeqnarray}
with $I_n$ as the pseudoscalar of the $n$-dimensional algebra, and the wedge symbol
$(\wedge)$ as the operator symbol of the outer product. The effect of the complement is to
turn basis elements not contained in the object $u$ into the constituting basis elements
of the complement of the object $\ubar{u}$ or $\bar{u}$. Connecting the object and its
complement with the wedge products creates an object that fills the available space in the
sense of requiring all basis elements to represent it. The full space is represented by
the pseudoscalar of that space, since the pseudoscalar $I_n$ is formed formed by the
product of all $n$ basis vectors of the space. \\

It can be shown that the left and right complements fulfill
\begin{equation}
    \bar{\ubar{a}} = a
    \label{eq:double_cmpl}
\end{equation}
which shows that the left and right complement operations are inverses to each other. This
is valid independent of the sequence of application, independent of the grade of the
argument $a$, and independent of the dimension of the space modelled in the algebra. Based
on this it is possible to define so-called regressive products using rules analog to
deMorgan's rules in boolean algebra as follows (definitions can be found
in~\cite{Lengyel_pga-illuminated:2024} and in \cite{Browne_Grassmann-Algebra_Vol1:2012}):
\begin{subeqnarray}
    % use \label for full label
    % use \slabel for sublabel
    a \vee b & \equiv & \underline{\bar{a} \wedge \bar{b}} \slabel{eq:rwdg_std} \\
    a \vee b & \equiv & \overline{\ubar{a} \wedge \ubar{b}} \slabel{eq:rwdg_alt}
\end{subeqnarray}
The same complement is applied to both arguments first, the resulting expressions are
combined by the wedge product, and afterwards the result uses the inverse complement
operation to transform the outcome of the wedge product into the final result of the
regressive product. The inverted wedge symbol $(\vee)$ is used for the regressive wedge
product. Both definitions~(\ref{eq:rwdg_std}) and~(\ref{eq:rwdg_alt}) are equivalent,
but~(\ref{eq:rwdg_std}) is used within the library to derive expressions for regressive
products. \\

There are some additional formulas for complements which are useful for working with GA
expressions and thus are provided here for reference. Taking the right and left
complements of~(\ref{eq:rwdg_std}) and~(\ref{eq:rwdg_alt}) we end up with
\begin{subeqnarray}
    % use \label for full label
    % use \slabel for sublabel
    \overline{a \vee b}  & = & \overline{\underline{\bar{a} \wedge \bar{b}}}
    \quad  =  \quad \bar{a} \wedge \bar{b}
    \slabel{eq:intermediate_std} \\
    \underline{a \vee b} & = & \underline{\ubar{a} \wedge \ubar{b}}
    \quad  =  \quad  \ubar{a} \wedge \ubar{b}
    \slabel{eq:intermediate_alt}
\end{subeqnarray}
Substituting $a = \ubar{u}$ and $b = \ubar{v}$ into~(\ref{eq:intermediate_std}) and $a =
\bar{u}$ and $b = \bar{v}$ into~(\ref{eq:intermediate_alt}) we end up with
\begin{subeqnarray}
    % use \label for full label
    % use \slabel for sublabel
    u \wedge v & = & \overline{\ubar{u} \vee \ubar{v}} \slabel{eq:wdg_std} \\
    u \wedge v & = & \underline{\bar{u} \vee \bar{v}} \slabel{eq:wdg_alt}
\end{subeqnarray}
This shows that the outer product and the regressive outer product can be expressed by
each other.

The complement operation is used to define a unique dualization operation that works for
cases when the metric is non-degenerate and even for cases when it is degenerate, like in
projective geometric algebra $PGA$. This is the main advantage compared to the usual
definition used by GA practitioners by multiplying the object with the pseudoscalar (or
its inverse), since this operation breaks down for spaces with degenerate metrics. The
right dual $A^{\star}$ is defined as $A^{\star} = \overline{GA}$, where $A$ is an
arbitrary multivector and $G$ the extended metric and $GA$ a matrix-vector-product. There
is also a corresponding left dualization operation defined as $A_{\star} =
\underline{GA}$. $\star$ is used as dualization operator (Hodge star). Relations to the
geometric product are $A^{\star} = A^{\dagger} \wedgedot I_n$ and $A_{\star} = I_n
\wedgedot A^{\dagger}$, where $\dagger$ is the reversion operation. \\


Following bi-argument products $\emph{op}(a,b) = a \; op \; b$ and their respective
regressive variants $\emph{rop}(a,b) = \underline{\bar{a} \; \emph{op} \; \bar{b}}$ are
implemented (with \path{op(a,b)} or \path{rop(a,b)} as the name of the implemented
function). Using $A$ vs. $a$ implies $gr(A) \ge gr(a)$, where $gr(arg)$ returns the grade
of the argument $arg$:
\begin{itemize}
    \item inner product, delivering a scalar: $a \bullet b$, \path{dot(a,b)}
    \item outer product of a $j-$ and $k-$vector, delivering a $j+k$-vector in
    $span(a,b)$: $a \wedge b$, \path{wdg(a,b)}
    \item geometric product:  $a \wedgedot b$, \path{operator*(a,b)}
    \item commutator product (the asymmetric part of the geometric product): $cmt(a,b) =
    \frac{1}{2}(a \wedgedot b - b \wedgedot a)$,
    \path{cmt(a,b)}
    \item regressive inner product, delivering a pseudoscalar: $a \circ b$,
    \path{rdot(a,b)}
    \item regressive outer product: $a \vee b$, \path{rwdg(a,b)}
    \item regressive geometric product: $a \veedot b$, \path{rgpr(a,b)}
    
    \item left (bulk) contraction: $a \lcontr B = a \ll B = a_{\star} \vee B $,
    \path{operator<<(a,B)}, \path{lbulk_contract(a,B)}  
    \item right (bulk) contraction: $B \rcontr a = B \gg a = B \vee a^{\star} $,
    \path{operator>>(B,a)}, \path{rbulk_contract(B,a)}

    \item left weight contraction: $a_{\smallstar} \vee B $,
    \path{lweight_contract(a,B)} 
    \item right weight contraction: $B \vee a^{\smallstar} $,
    \path{rweight_contract(B,a)} 

    \item left bulk expansion: $A_{\star} \wedge b $,
    \path{lbulk_expand(A,b)} 
    \item right bulk expansion: $a \wedge B^{\star} $,
    \path{rbulk_expand(a,B)} 

    \item left weight expansion: $A_{\smallstar} \wedge b $,
    \path{lweight_expand(A,b)} 
    \item right bulk expansion: $a \wedge B^{\smallstar} $,
    \path{rweight_expand(a,B)} 
\end{itemize}
where the items from the left contraction downwards are all considered interior products.
They operate on subspaces after applying a dualization operation to one operand. \\


TODO: \emph{fill in further introductory notes here} \\

\newpage
