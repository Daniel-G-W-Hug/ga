\subsection{Introduction}
\label{intro}

The ga library is intended to make Geometric Algebra more accessible by providing
functionality for numerical calculations required for simulation of physical systems.
\\

It implements functionality for two- and three-dimensional spaces with Euclidean geometry.
Currently it provides data types and operations to handle the algebras of regular
Euclidean geometry $(ega2d, ega3d)$ and projective Euclidean geometry $(pga2dp, pga3dp)$.
The library handles multivectors, even and uneven grade multivectors as well as their
components like scalars, vectors, bivectors, trivectors, and pseudoscalars of the
corresponding linear spaces efficiently. It can be easily extended to provide
functionality for handling spacetime algebra $(STA)$ with limited effort. \\

The main library is a header only library in the ga folder which can be used by including
either
\begin{verbatim}
    #include ga/ga_ega.hpp // or
    #include ga/ga_pga.hpp
\end{verbatim}
and by making the corresponding namespaces accessible for use by stating
\begin{verbatim}
    using namespace hd::ga;      // and either
    using namespace hd::ga::ega; // or
    using namespace hd::ga::pga;
\end{verbatim}
in your application code. For a simple usage example please have a look either at
\path{ga_test/src/ga_ega_test.cpp} or at \path{ga_test/src/ga_pga_test.cpp}.
\\

All product expressions are generated via the file \path{ga_prdxpr/ga_prdxpr_main.cpp} and
user input provided in the header files of the corresponding algebra in that folder. The
generated coefficient expressions are used to fill-in the source code for geometric
products, wedge and dot products or their regressive counterparts. If certain combinations
are not yet implemented in the ga library they can easily be generated within
\path{ga_prdxpr} as needed and then be added to the library implementation. \\

The complement operation used in this library corresponds to the complement as defined in
\cite{Lengyel_pga-illuminated:2024}. It is uniquely determined with respect to the outer
product (not with respect to the geometric product as commonly used by many autors working
with geometric algebra). This is helpful to generate consistent signs of expressions,
operators, complements and duals. The left complement $\ubar{u}$ is defined as $\ubar{u}
\wedge u = I_n$ and the right complement $\bar{u}$ as $u \wedge \bar{u} = I_n$ with $I_n$
as the pseudoscalar of the $n$-dimensional space of the algebra. \\

The complement operation is used in turn to define a unique dualization operation that
works for cases when the metric is non-degenerate as well as for cases where it is
degenerate, like in projective geometric algebra. The right dual $A^{\star}$ is defined as
$A^{\star} = \overline{GA}$, where $A$ is an arbitrary multivector and $G$ the extended
metric. There is also a corresponding left dualization operation defined as $A_{\star} =
\underline{GA}$. $\star$ is used as dualization operator (Hodge star). Relations to the
geometric product $A^{\star} = A^{\dagger} \wedgedot I_n$ and $A_{\star} = I_n \wedgedot
A^{\dagger}$ also exist with $\dagger$ as the reversion operation. \\


Following products $\emph{op}(a,b)$ and their respective regressive variants
$\emph{rop}(a,b) = \underline{\bar{a} \; \emph{op} \; \bar{b}}$ are implemented:
\begin{itemize}
    \item inner product, delivering a scalar: $a \bullet b$, \path{dot(a,b)}
    \item outer product of a $j-$ and $k-$vector, delivering a $j+k$-vector in
    $span(a,b)$: $a \wedge b$, \path{wdg(a,b)}
    \item geometric product:  $a \wedgedot b$, \path{operator*(a,b)}
    \item commutator product (the asymmetric part of the geometric product): $cmt(a,b) =
    \frac{1}{2}(a \wedgedot b - b \wedgedot a)$,
    \path{cmt(a,b)}
    \item regressive inner product, delivering a pseudoscalar: $a \circ b$,
    \path{rdot(a,b)}
    \item regressive outer product: $a \vee b$, \path{rwdg(a,b)}
    \item regressive geometric product: $a \veedot b$, \path{rgpr(a,b)}
    
    \item left (bulk) contraction: $a \lcontr B$ or $a \ll B = a_{\star} \vee B $,
    \path{operator<<(a,B)}, \path{lbulk_contract(a,B)}  
    \item right (bulk) contraction: $B \rcontr a$ or $B \gg a = B \vee a^{\star} $,
    \path{operator>>(B,a)}, \path{rbulk_contract(B,a)}

    \item left weight contraction: $a_{\smallstar} \vee B $,
    \path{lweight_contract(a,B)} 
    \item right weight contraction: $B \vee a^{\smallstar} $,
    \path{rweight_contract(B,a)} 

    \item left bulk expansion: $A_{\star} \wedge b $,
    \path{lbulk_expand(A,b)} 
    \item right bulk expansion: $a \wedge B^{\star} $,
    \path{rbulk_expand(a,B)} 

    \item left weight expansion: $A_{\smallstar} \wedge b $,
    \path{lweight_expand(A,b)} 
    \item right bulk expansion: $a \wedge B^{\smallstar} $,
    \path{rweight_expand(a,B)} 
\end{itemize}
where the items from the left contraction downwards are all considered interior products.
If an operand is written above as capital letter, the grade of the object must be higher
than the grade of the other operand for a meaningful result. \\


TODO: \emph{fill in further introductory notes here} \\

\newpage
