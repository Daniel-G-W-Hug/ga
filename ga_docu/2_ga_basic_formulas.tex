\subsection{Basic formulas}
\label{basic_formulas}

The inner product between multivectors $A$ and $B$ is defined as
\begin{equation}
    A \bullet B = A^{T} G B
    \label{eg:inner_product_metric}
\end{equation}
with $G$ as the extended metric and matrix multiplication on the right hand side. It
satisfies the identity
\begin{equation}
    A \bullet B  = \grprj{A \tilde{B}}{0} = \grprj{ B \tilde{A} }{0}
    \label{eq:inner_product_link_to_gpr}
\end{equation}
where juxtaposition within the grade projection operator stands for the geometric product.
The norm of a multivector $A$ is defined as
\begin{equation}
    \nrm{A} = \sqrt{A \bullet A}
    \label{eq:norm_mv}
\end{equation}
with a positive argument under the square root due to the definition of the inner product.
\\

The contraction is explicitly defined as
\begin{subequations}
    \begin{align}
    A \lcontr B & = A_{\star} \vee B
    \label{eq:lcontr} \\
    A \rcontr B & = A \vee B^{\star}
    \label{eq:rcontr}
    \end{align}
\end{subequations}
with $\star$ denoting the Hodge dual with is formed by the respective complement operation
(in spaces of even dimension the left or right complement respectively, and in spaces of
uneven dimension the complement function regardless of the side of the operand). For
blades A and B, they satisfy
\begin{subequations}
    \begin{align}
    A \lcontr B & = \grprj{B \tilde{A}}{gr(B)-gr(A)}
    \label{eq:lcontr_gpr} \\
    A \rcontr B & = \grprj{\tilde{B} A}{gr(A)-gr(B)}
    \label{eq:rcontr_gpr}
    \end{align}
\end{subequations}
and fulfill the requirement that $A \lcontr B = A \rcontr B = A \bullet B$ whenever $A$
and $B$ have the same grade. In this case the contractions reduce to the inner product. \\


The geometric product between and vector $v$ and a blade $B$ is defined in terms of
the inner and outer products, i.e. the right contraction $\rcontr$, the left contraction
$\lcontr$ and wedge product $\wedge$ as follows:
\begin{subequations}
    \begin{align}
    a \wedgedot B & =  B \rcontr a + a \wedge B
    \label{eq:gpr_Br} \\
    B \wedgedot a & =  a \lcontr B + B \wedge a 
    \label{eq:gpr_Bl}
    \end{align}
\end{subequations}
$a = a_{\parall} + a_{\perp}$ can be decomposed into parts $a_{\parall}$ parallel to the
$k$-blade $B$ and $a_{\perp}$ perpendicular to $B$. For equation {(\ref{eq:gpr_Br})}: $B
\rcontr a$ is a $(k-1)$-blade. If $a_{\parall} \neq 0$ then $B \rcontr a$ represents a
subspace of $B$. $a \wedge B$ is a $(k+1)$-blade. If $a_{\perp} \neq 0$ then $a \wedge B$
represents span$(a,B)$.

For arguments of equal grade the contractions reduce to the dot
product $\cdot$, so that one can write specifically for two vectors $a$ and $b$:
\begin{equation}
    a \wedgedot b =  a \cdot b + a \wedge b
\end{equation}

Text with a norm $\nrm{x}$ and an indexed norm as $\bulk{\nrm{u}}$ and
$\weight{\nrm{u}}$.\\


TODO: \emph{fill in basic formulas here}

\newpage