\subsection{Basic formulas}
\label{basic_formulas}

The following is provided for Euclidean algebra of two-dimensional space (\emph{ega2d})
using an orthormal basis \bv{1}, \bv{2}:
\begin{subequations}
    \begin{align}
        (\bv{1})^2 & = \bv{1}^2 = +1, \;\text{and}\; (\bv{2})^2 = \bv{2}^2 = +1
        \label{eq:ega2d_base_vec} \\
        s_{2d} & = s\bm{1}
        \label{eq:scalar2d} \\
        v_{2d} & = v_1\bv{1} + v_2\bv{2} 
        \label{eq:vec2d} \\
        ps_{2d} & = ps \bv{12}  = ps\mathds{1}
        \label{eq:pscalar2d}
    \end{align}
\end{subequations}
equation (\ref{eq:scalar2d}) is the scalar part $s_{2d}$, equation (\ref{eq:vec2d})
contains the vector part $v_{2d}$, and equation (\ref{eq:pscalar2d}) contains the
pseudoscalar part $ps_{2d}$. The index $2d$ is typically omitted when clear from context.
The basis elements are $\{\bm{1}, \bv{1}, \bv{2}, \bv{12}\}$. Using these components a
multivector $M$ of $2d$ space is defined as
\begin{equation}
     M = s \bm{1} + v_1\bv{1} + v_2\bv{2} + ps\bv{12}
    \label{eq:mvec2d}   
\end{equation}
where equation (\ref{eq:mvec2d}) contains three parts: the scalar part $s\bm{1}$ (basis
element is the scalar \textbf{1}; if \textbf{1} is not shown, it is implicitly assumed for
scalar values), the vector part $v$ (basis elements \bv{1} and \bv{2}) and the
pseudoscalar part $ps \bv{12} = ps \mathds{1}$ (basis element is \bv{12}, which is
sometimes written as $\mathds{1}$ to show its character as pseudoscalar of this space. It
is a bivector in $2d$-Euclidean space). \\

For Euclidean algebra of three-dimensional space (\emph{ega3d}) using an orthonormal basis
\bv{1}, \bv{2}, \bv{3} there are:
\begin{subequations}
    \begin{align}
        (\bv{1})^2 = \bv{1}^2 & = +1, (\bv{2})^2 = \bv{2}^2 = +1,
        \;\text{and}\; (\bv{3})^2 = \bv{3}^2 = +1
        \label{eq:ega3d_base_vec} \\ 
        s_{3d} & = s\bm{1}
        \label{eq:scalar3d} \\
        v_{3d} & = v_1\bv{1} + v_2\bv{2} + v_3\bv{3} 
        \label{eq:vec3d} \\ 
        b_{3d} & = b_1\bv{23} + b_2\bv{31} + b_3\bv{12} 
        \label{eq:bivec3d} \\ 
        ps_{3d} & = ps \bv{123}  = ps\mathds{1}
        \label{eq:pscalar3d}
    \end{align}
\end{subequations}
equation (\ref{eq:scalar3d}) is the scalar part $s_{3d}$, equation (\ref{eq:vec3d}) is the
vector part $v_{3d}$, equation (\ref{eq:bivec3d}) is the bivector part $b_{3d}$, and
equation (\ref{eq:pscalar3d}) is the pseudoscalar part $ps_{3d}$. The index $3d$ is
typically omitted when clear from context. Comparing to the $2d$-case it becomes obvious,
that all parts depend on context, specifically on the dimensionality of the modeled space,
and thus need to be defined and treated accordingly (\emph{hint}: since \Cpp is a
statically typed language those types need to be well-defined and distinguishable from
each other). The basis elements are $\{\bm{1}, \bv{1}, \bv{2}, \bv{3}, \bv{23}, \bv{31},
\bv{12}, \bv{123}\}$. Using these components a multivector $M$ of $3d$ space is defined as
\begin{equation}
    M = s \bm{1} + v_1\bv{1} + v_2\bv{2} + v_3\bv{3} 
    + b_1\bv{23} + b_2\bv{31} + b_3\bv{12} + ps\bv{123}
    \label{eq:mvec3d}  
\end{equation}
where equation (\ref{eq:mvec3d}) contains four parts: the scalar component $s\bm{1}$
(basis element is the scalar \textbf{1}, the vector part $v$ (basis elements \bv{1},
\bv{2} and \bv{3}), the bivector part $b$ (basis elements \bv{23}, \bv{31} and \bv{12})
and the pseudoscalar part $ps \bv{123} = ps \mathds{1}$ (basis element is \bv{123}, which
is sometimes written as $\mathds{1}$ to show its character as pseudoscalar of this space.
It is a trivector in $3d$-Euclidean space). \\

TODO: \emph{fill in basic product definitions of ega2d and ega3d here} \\

TODO: \emph{fill in pga2d and pga3d definitions here} \\


The inner product between multivectors $A$ and $B$ is defined as
\begin{equation}
    A \bullet B = A^{T} G B
    \label{eg:inner_product_metric}
\end{equation}
with $G$ as the extended metric and matrix multiplication on the right hand side. It
satisfies the identity
\begin{equation}
    A \bullet B  = \grprj{A \wedgedot \tilde{B}}{0} = \grprj{ B \wedgedot \tilde{A} }{0}
    \label{eq:inner_product_link_to_gpr}
\end{equation}
where $\wedgedot$ stands for the geometric product within the grade projection operator
\grprj{}{k} for grade $k$. The norm of a multivector $A$ is defined as
\begin{equation}
    \nrm{A} = \sqrt{A \bullet A}
    \label{eq:norm_mv}
\end{equation}
with a positive argument under the square root due to the definition of the inner product.
\\

The contraction is explicitly defined as
\begin{subequations}
    \begin{align}
    A \lcontr B & = A \ll B = A_{\star} \vee B
    \label{eq:lcontr} \\
    B \rcontr A & = B \gg A = B \vee A^{\star}
    \label{eq:rcontr}
    \end{align}
\end{subequations}
with $\star$ denoting the Hodge dual which is formed by the respective complement
operation (in spaces of even dimension the left or right complement respectively, and in
spaces of uneven dimension the complement function regardless of the side of the operand).
For blades A and B, they satisfy
\begin{subequations}
    \begin{align}
    A \lcontr B &  = A \ll B = \grprj{B \wedgedot \tilde{A}}{gr(B)-gr(A)}
    \label{eq:lcontr_gpr} \\
    B \rcontr A & = B \gg A = \grprj{\tilde{A} \wedgedot B}{gr(B)-gr(A)}
    \label{eq:rcontr_gpr}
    \end{align}
\end{subequations}
and fulfill the requirement that $A \lcontr B = A \rcontr B = A \bullet B$ whenever $A$
and $B$ have the same grade. In this case the contractions reduce to the inner product. \\


The geometric product between and vector $v$ and a blade $B$ is defined in terms of
interior and exterior products, i.e. the right contraction $\rcontr$, the left contraction
$\lcontr$ and wedge product $\wedge$ as follows:
\begin{subequations}
    \begin{align}
    a \wedgedot B & =  B \rcontr a + a \wedge B
    \label{eq:gpr_Brc} \\
    B \wedgedot a & =  a \lcontr B + B \wedge a
    \label{eq:gpr_Blc}
    \end{align}
\end{subequations}
or sometimes using the right shift ($\gg$) or left shift ($\ll$) operators directly
\begin{subequations}
    \begin{align}
    a \wedgedot B & = B \gg a + a \wedge B
    \label{eq:gpr_Bshrc} \\
    B \wedgedot a & =  a \ll B + B \wedge a
    \label{eq:gpr_Bshlc}
    \end{align}
\end{subequations}
as an alternative. \\

Every vector $a$ can be decomposed into parts $a_{\parall}$ parallel to the $k$-blade $B$
and $a_{\perp}$ perpendicular to $B$, such that $a = a_{\parall} + a_{\perp}$. For
equation {(\ref{eq:gpr_Bshrc})}: $B \gg a$ is a $(k-1)$-blade. If $a_{\parall} \neq 0$
then $B \gg a$ represents a subspace of $B$. $a \wedge B$ is a $(k+1)$-blade. If
$a_{\perp} \neq 0$ then $a \wedge B$ represents $span(a,B)$.

For arguments of equal grade the contractions reduce to the dot
product $\bullet$, so that one can write specifically for two vectors $a$ and $b$:
\begin{equation}
    a \wedgedot b =  a \bullet b + a \wedge b
\end{equation}

TODO: \emph{fill in further basic formulas here} \\

% Text with a norm $\nrm{x}$ and an indexed norm as $\bulk{\nrm{u}}$ and
% $\weight{\nrm{u}}$.\\
Text with a norm $\nrm{x}$ and an indexed norm as $\bulknrm{u}$ and
$\weightnrm{u}$.\\

$\bm{e}_1$, \bv{1}, $\bv{123}$, $\filledstar$, $\star$, $\smallstar$
\\




\newpage